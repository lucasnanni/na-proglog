\documentclass[a4paper,11pt]{article}

\usepackage[a4paper,margin=1.5cm]{geometry}
\usepackage{parskip}
\usepackage[utf8]{inputenc}
\usepackage[brazil]{babel}

\usepackage{lmodern}
\usepackage[scale=2]{ccicons}
\usepackage{hyperref}

\usepackage[center]{titlesec}
\usepackage{enumitem}

\usepackage{color}
\usepackage{fancyvrb}
\usepackage{comment}
\excludecomment{verbatim}

% Usado na saída de tools/filter
\newcommand{\mylambda}{$$\lambda$$}

\setcounter{secnumdepth}{0}

\setlist[enumerate,1]{label*=\textbf{$num$.\arabic*{)}}}

\hypersetup{
    unicode,
    pdfauthor={Marco A L Barbosa},
    pdftitle={Paradigma de Programação Lógico - $title$ - Exercícios},
}

$for(header-includes)$
$header-includes$
$endfor$

\begin{document}

\pagestyle{empty}

\section{Paradigma de Programação Lógico}

\section{$num$ - $title$ - Exercícios}

$body$

\section{Licença}

\begin{center}
Os exercícios sem referências são de autoria de \href{malbarbo.pro.br}{Marco A
L Barbosa} e estão licenciados com a Licença Creative Commons -
Atribuição-CompartilhaIgual 4.0 Internacional.

\href{http://creativecommons.org/licenses/by-sa/4.0/}{\ccbysa}
\end{center}

\end{document}

% vim: set spell spelllang=pt:
